\documentclass[a4paper,12pt]{report}

\usepackage[english]{babel} %language selection
\selectlanguage{english}

%\usepackage{html,makeidx}
\usepackage{pslatex}
\usepackage{boxedminipage}
\usepackage{listings}

\usepackage{titlesec}
\titleformat{\chapter}[hang]{\bf\huge}{\thechapter}{2pc}{}

\pagenumbering{arabic}

\usepackage{hyperref}
\hypersetup{colorlinks, 
           citecolor=black,
           filecolor=black,
           linkcolor=black,
           urlcolor=black,
           bookmarksopen=true}
           
\usepackage{html}   %  *always* load this for LaTeX2HTML
\begin{htmlonly}
  \usepackage{verbatim}
  \providecommand{\lstinputlisting}[2][]{\verbatiminput{#2}}
\end{htmlonly}


%\hfuzz = .6pt % avoid black boxes

\newcommand{\devvar}{The default value is suitable and this variable is made configurable mostly for developmental purposes}

% style for XML Config listings
\lstdefinestyle{xmlconfig}{language=XML,basicstyle=\small\ttfamily,numbers=left,frame=lines,breaklines=true}

% style for C Code listings
\lstdefinestyle{c_code}{language=C,basicstyle=\small\ttfamily,numbers=left,frame=lines,breaklines=true}

\begin{document}

    \pagestyle{empty}
    \vspace*{15cm}
    \begin{flushright}
        \rule[-1ex]{11cm}{3pt}\\
        \huge Performance Model Manager\\
        \huge \emph{User Manual}\\
    \end{flushright}

    \newpage


    \noindent

    \begin{boxedminipage}[b]{\textwidth}
    Performance Model Manager User Manual,
    Copyright \copyright{} 2010 Robert Higgins.
    Permission is granted to copy, distribute and/or modify this document
    under the \\ terms of the GNU Free Documentation License, Version 1.3
    or any later version published by the Free Software Foundation;
    with no Invariant Sections, no Front-Cover Texts, and no Back-Cover Texts.
    A copy of the license is included in the section entitled ``GNU
    Free Documentation License''.
    \end{boxedminipage}

    \newpage

    \tableofcontents


    \chapter{Introduction}

    Performance Model Manager (PMM) is an open source GNU Public Licenced tool
    for experimentation in the use of Functional Performance Models (FPMs). An
    FPM describes the speed of a computational routine in terms of the
    routine's input parameters. It focuses on addressing issues surrounding the
    construction, maintenance and use of FPMs. To this end, it has three main
    features:
    \begin{itemize}
        \item It implements the Geometric Bisection Building Procedure for
            multi-parameter FPMs, optimizing the construction of a problem's
            performance model. 
        \item It permits the construction of models for a large number of
            problems by implementing a flexible benchmarking scheduler.
        \item It provides access to the models in a variety of ways, so that
            they may be visualised or used to make scheduling decisions.
    \end{itemize}
    Construction of the FPM of some general computational procedure is supported
    by requiring that the user provides a benchmarking executable which behaves
    according to a simple protocol.
    
    An example implementation of such a 'benchmark binary' is given in Chapter
    \ref{chap:writingbench} along with details of its configuration within PMM.
    Further examples are also included in the source distribution.

    Models can be constructed on demand or in the background by the \verb+pmmd+
    daemon. Construction of multiple models can be scheduled according to a
    variety of policies. They will be constructed in turn according to their
    priority and scheduling criteria.

    Access to models will be made available via an API in a future release of
    PMM, at present only viewing of models is possible, via a plotting program:
    \verb+pmm_view+.

    This manual continues to Chapter \ref{install_chap} where compilation and 
    installation is described. Then the PMM configuration file is described in
    Chapter \ref{config_chap}. Chapter \ref{bench_chap} provides notes on how
    to write a benchmark binary and configure it as a 'routine' to be modelled
    within
    PMM.

    \chapter{Installation}
    \label{install_chap}

    \section{Requirements}
    PMM is developed for the Linux platform but may also compile on other POSIX
    operating systems. The following softwares are required to install PMM:
    \begin{itemize}
        \item Gnuplot
        \item GNU Make
        \item GCC compiler suite (tested with 4.x series only)
        \item libxml2 2.6.0 or greater
        \item GNU Scientific Library
    \end{itemize}
    The following are optional but enable certain features:
    \begin{itemize}
        \item Octave (2.9.14 or greater) is required for multi-parameter model construction
        \item PAPI (4.0.0 tested) is required for higher resolution timing and automatic complexity calculation
        \item GotoBLAS2 is required for for further example problems
    \end{itemize}

    \section{Compiling \& Installing}
    Installation of PMM uses a hierarchy of directories under a certain prefix,
    by default \verb+/usr+. If this is not desirable the build should be
    configured with the \verb+--prefix=<dir>+ option. A typical installation
    follows:

    \begin{verbatim}
        $ tar -xzf pmm-0.0.1.tar.gz
        $ cd pmm-0.0.1
        $ ./configure --prefix=$PWD/install
        $ make && make install
    \end{verbatim}

    \noindent Configuration options to note:

    \begin{itemize}
        \item \verb+--enable-debug+ enable debugging messages and flags
        \item \verb+--disable-octave+ disable use of octave and multi-parameter
            model support
        \item \verb+--disable-benchgslblas+ disable compilation and
            installation of demonstration GSL problem benchmarks
        \item \verb+--enable-benchgotoblas2+ enable compilation and
            installation of demonstration GotoBLAS2 problems
        \item \verb+--with-papi[=path]+ enable use of PAPI with optional
            specification of PAPI installation path
    \end{itemize}

    \noindent Further options can be viewed by running \verb+./configure --help+
    .

    After installation, the PMM daemon is started by executing the
    \verb+pmmd+ binary and the PMM viewer program is run via the \verb+pmm_view+
    binary.


    \chapter{Configuration}
    \label{config_chap}
    PMM is distributed with a default configuration which will be installed under:
    \begin{verbatim} <prefix>/etc/pmmd.conf[.sample] \end{verbatim}

    \noindent This can serve as a template for a user's own configuration and
    contains sane values for all options. If example problem benchmarks are
    built, the sample configuration will also describe those problems.
    
    The configuration file has an heirarchical XML structure Configuration is
    described between \verb+<config>+ root element tags.  Under this, the load
    monitor facility is described by a \verb+<load_monitor>+ element, and each
    routine for which a model is to be built, is described by a 
    \verb+<routine>+ element.

    In the following sections, each element in the configuration file is
    described. If an element has a default value, it need not be explicitly set
    in the configuration file, on the other hand, some options must be set.
    This information, along with the type of the expected element value
    (string, integer, etc.) and what exactly the element describes is detailed
    below.

    \section{General Configuration}
    The following elements (which can be seen in context in Listing
    \ref{basic_config_example}) define some general application configurable
    options and come directly under the \verb+<config>+ tags:
    \begin{itemize}
        \item \verb+<main_sleep_period>+ (\emph{integer, default:1}) The
            benchmark scheduler checks the system state ever $n$ seconds and
            this period may be configured here. \devvar.
        \item \verb+<model_write_time_threshold>+ (\emph{integer, default:60})
            When benchmarking problems of very small size, which execute very
            quickly, the manager may become overloaded by writing the model to
            disk after each execution. This option allows us to configure how
            often the model will be saved to disk, i.e. after a total of $n$
            seconds has been spent benchmarking a particular model it will be
            written to disk.  \devvar.
        \item \verb+<model_write_execs_threshold>+ (\emph{integer, default:10})
            This option serves the same purpose as the previous one, except
            that it specifies the number of benchmark executions that must
            occur before it is written to disk.  It may take hundreds of small
            benchmarks exceed the time threshold (above), so this second
            threshold allows us to write based on execution frequency as well.
            \devvar.
    \end{itemize}

    \begin{lstlisting}[style=xmlconfig,caption=Basic Configuration,float=h,label=basic_config_example]
<?xml version="1.0"?>
<config>
   <main_sleep_period>1</main_sleep_period>
   <model_write_time_threshold>60
      </model_write_time_threshold>
   <model_write_execs_threshold>20
      </model_write_execs_threshold>

   <load_monitor>
      <load_path>/usr/var/pmm/loadhistory</load_path>
      <write_period>60</write_period>
      <history_size>60</history_size>
   </load_monitor>

   ....
</config>
    \end{lstlisting}

    \section{Load Monitor Configuration}

    The load monitoring facility is is described by a \verb+<load_monitor>+
    element, this has the following children:

    \begin{itemize}
        \item \verb+<load_path>+ (\emph{string, required}) path to a file where
            load observations are recorded
        \item \verb+<write_period>+ (\emph{integer, default:360}) frequency
            with which to save the load file to disk (in seconds)
        \item \verb+<history_size>+ (\emph{integer, default:60}) number of load
            observations to store \end{itemize}


    \section{Routine Configuration}

    \noindent Each routine is described by a \verb+<routine>+ element.
    Routines have detailed descriptions of the parameters to be passed to them
    and the construction method that should be used to build their models. An
    example routine configuration can been seen in Listing
    \ref{routine_config_example}, it describes a 2-parameter routine. First,
    the general options are set using the following child elements:
    \begin{itemize}
        \item \verb+<name>+ (\emph{string, required}) The routine name.
        \item \verb+<exe_path>+ (\emph{string, required}) The path to the
            benchmarking executable
        \item \verb+<model_path>+ (\emph{string, required}) The path to the
            file where the performance model of the routine will be saved.
        \item \verb+<priority>+ (\emph{integer, default:0}) The construction
            priority this routine has (logically, the higher the value, the
            higher the priority)
    \end{itemize}

    \noindent The parameters of a routine are described by a 
    \verb+<parameters>+ elmenet. These are the parameters that will be passed to
    the benchmark binary which ultimately executes the routine which is being
    modeled. The parameters passed to the benchmark are those that influnce the
    volume of computations or the speed at which the computations are carried
    outt. This is further described in Chapter \ref{bench_chap}. The first
    child of the \verb+<parameters>+ element must be the number of parameter
    descriptions which will follow:
    \begin{itemize}
        \item \verb+<n_p>+ (\emph{integer, required}) number of parameters
            which the benchmark accepts
    \end{itemize}
    Following that, each parameter is described by a \verb+<param>+ element.
    The \verb+<param>+ element has a number of child elements which are:
    \begin{itemize}
        \item \verb+<order>+ (\emph{integer, required}) The order
            in which this parameter should be passed to the benchmark binary.
        \item \verb+<name>+ (\emph{string, required}) The name of this
            parameter.
        \item \verb+<min>+ (\emph{integer, required}) The
            minimum value this parameter may have. If modelling the performance
            of the processor while operating in cache only is \emph{not}
            important, this should be set so the overall problem size is large
            enough to occupy main memory.
        \item \verb+<max>+ (\emph{integer, required}) The maximum value this
            parameter may have. This should be large enough to induce
            significant paging.
        \item \verb+<stride>+ (\emph{integer, default:1}) The stride with which
            this parameter should be incremented. Stride influences the
            climbing phase of optimised construction (where successive
            benchmarks are incremented in size by this value) as well as naive
            construction (where all points on the stride between min and max
            are benchmarked). A reasonable value for stride would be, for
            example, 1/100th of the range between max and min. If stride is too
            low, excessive time may be spent building a model.
        \item \verb+<offset>+ (\emph{integer, default:0}) Offset for this
            parameter. If required, you can specify that a parameter value must
            always be a certain offset from zero.
        \item \verb+<fuzzy_max>+ (\emph{boolean, default:false}) This specifies
            that the maximum parameter size defined is not a true max and speed
            at this maximum should be measured.
            
            In normal circumstances the FPM is constructed across
            a complete range of problem sizes, from small to so large that
            speed is effectively zero. The maximum parameter value will be so
            large that it induces heavy paging. Speed at this maximum is not
            measured, but assumed to be zero. If this is \emph{not} the case,
            and the maximum parameter size will not induce heavy paging,
            \verb+<fuzzy_max>+ must be set to \emph{true} for the GBBP algorithm
            to complete successfully.
    \end{itemize}


    \noindent Directives for the construction method must be described by a
    \verb+<construction>+ element. It has the following child elements:
    \begin{itemize}
        \item \verb+<method>+ (\emph{string, default:gbbp}) The construction
            method, this element may have the following values:
        \begin{itemize}
            \item \emph{gbbp} - the Geometric Bisection Building Procedure will
                be used to select benchmark points, minimising the number of
                points required to accurately estimate the model
            \item \emph{naive} - all possible points between the parameter
                ranges will be benchmarked.
            \item \emph{rand} - points between the parameter ranges will be
                selected at random
        \end{itemize}
        \item \verb+<min_sample_num>+ (\emph{integer, default:1}) Specify the
            minimum number of benchmarks to be taken at a single point in the
            model. Once this is met, the point will be considered as measured
            and the construction method will proceed to the next point of its
            choosing.
        \item \verb+<min_sample_time>+ (\emph{integer, default:0}) Specify the
            minimum number of seconds that should be spent in the benchmarking
            of a single point before it is considered as measured. I.e. if set
            to 60 seconds, a benchmark taking 20 seconds will be measured 3
            times.
    \end{itemize}

    Finally, priority and scheduling policy may be specified. When multiple
    routines are configured in PMM priorities allow the user to specify which
    models will be built first. Scheduling policies allow the user limit the
    execution of benchmarks to certain time periods or certain system
    conditions.

    \begin{itemize}
        \item \verb+<priority>+ (\emph{integer, default:0}) Priority of
            construction for the routine. Higher priority routines will have
            their models constructed before lower ones
        \item \verb+<condition>+ (\emph{string, default:now}) Condition under
            which benchmarking of a routine is permitted. Note: Once started, a
            benchmark will not be interrupted, even if the conditions that
            permitted its execution have changed to ones which would otherwise
            prevent execution.
            \begin{itemize}
                \item \emph{now} - construction is permitted at all times
                \item \emph{idle} - construction is only permitted when the
                    observed 5 minute load average is less than 0.10 (note: the
                    act of benchmarking will influence the load average of the
                    system. After the benchmark is complete, PMM will probably
                    have to wait ~5 minutes before the next execution can
                    occur)
                \item \emph{nousers} - construction is only permitted when no
                    users are logged into the system. Logged in users would be
                    those reported by utilities such as \verb+w+, \verb+who+,
                    \verb+users+ and so on.
            \end{itemize}
    \end{itemize}


    \begin{lstlisting}[style=xmlconfig,caption=Routine Configuration Example,label=routine_config_example]
<routine>
   <name>dgemm2</name>
   <exe_path>/usr/local/lib/pmm/dgemm2</exe_path>
   <model_path>/usr/local/var/pmm/dgemm2_model</model_path>
   <parameters>
      <n_p>2</n_p>
      <param>
         <order>0</order>
         <name>m</name>
         <min>32</min>
         <max>4096</max>
         <stride>32</stride>
         <offset>0</offset>
      </param>
      <param>
         <order>1</order>
         <name>n</name>
         <min>32</min>
         <max>4096</max>
         <stride>32</stride>
         <offset>0</offset>
      </param>
   </parameters>
   <construction>
      <method>gbbp</method>
      <min_sample_num>5</min_sample_num>
      <min_sample_time>120</min_sample_time>
   </construction>
   <condition>now</condition>
   <priority>30</priority>
</routine>
    \end{lstlisting}

    \chapter{Building the FPM of a Computation}
    \label{bench_chap}
    This chapter outlines what a user must do to have PMM build the FPM of some
    computation. The computation may be a library subroutine, a code
    fragment or an entire process. Throughout this document this computation
    will be referred to as a \emph{routine}. The users routine must be wrapped
    in a benchmarking binary or script which should behave in a specific
    manner:

    \begin{itemize}
        \item It must accept arguments from the command line which define
            the volume of computations it must carry out.
        \item It must execute and time the computation that is to be modeled.
            Execution may be via a script or compiled binary, written in any
            language, and the details of how it perform or times the
            computation do not concern the PMM tool. If the benchmark is written
            in C/C++, PMM provides some utilities to aid this in a shared
            library, \verb+libpmm+.
        \item It must output timing and volume of computations (complexity) in
            a standard manner. \verb+libpmm+ also supports this.
    \end{itemize}

    Implementation of this benchmark is a task left to the user. The following
    sections describe how to choose input parameters, write the benchmark and
    configure PMM to build a FPM for the routine.

    \section{Choosing Parameters of a Routine}

    The first step a user must take is to identify the parameters of the
    routine which effect the volume of computations it must carry out.
    Typically, the volume of computations would be floating point operation
    count, however PMM is agnostic to the type of computations the routine
    carries out, and the volume may be expressed as the user wishes.
    The performance model that we build will be expressed in terms of these
    parameters. Throughout this chapter we will refer to an example of a square
    matrix multiplication. In this scenario, there is only one parameter that
    effects the volume of computations, $N$, the length of a matrix side in the
    multiplication.

    For a more general case, were two matrices of sizes $N \times K$ and $K
    \times M$ are multiplied and the result stored in an third matrix of size
    $N \times M$, then the volume of computations would depend on three
    parameters, $N$, $M$ and $K$.

    A general purpose matrix multiplication routine usually has other
    associated parameters defining transpositions of the input data and other
    coefficients. These however do not contribute significantly to the
    computational complexity of the routine and they should not be considered
    as parameters of the model in the PMM framework. It is important to note
    than building an FPM which is in terms of more than one parameter is very
    intensive as the number of points required to accurately approximate scales
    exponentially with the number of parameters the model is in terms of. Any
    parameters of a routine that can be excluded from the functional
    performance model should be.

    \section{Writing a Benchmark for PMM} \label{chap:writingbench}

    For PMM to build the performance model of a routine, it must be able to
    execute benchmarks of that routine for various problem sizes. As previously
    stated, the problem's size is determined by the parameters which effect the
    computational complexity of the routine, and the performance model is a
    function of these parameters.
    
    PMM must be provided with an executable which carries out a benchmark with
    given input parameters.  The user must write this executable so that it
    behaves in a specified way. PMM is distributed with the source of a number
    of example benchmarks, here we will list one and reference it as the
    required behaviours are described below. Listing \ref{square_mxm_code}
    shows an example benchmark for a square matrix multiplication routine. The
    multiplication is provided by the Gnu Scientific Library. Note in a square
    matrix multiplication, the volume of computations is determined by the size
    of one side of matrices to be multiplied.

    The benchmark code behaves in the following manner:

    \begin{itemize}
        \item The executable must accept a number of parameters on the command
            line. These parameters will also be described in the configuration
            entry for the routine. As of version 0.0.1 parameters can only have
            integer types. (Lines 17-23)
        \item Based on the parameters passed on the command line, the benchmark
            must initialise memory and data structures that are to be passed to
            the routine. If the computation that is to be modelled is just a
            simple code fragment, no allocation of memory that occurs within
            the code fragment should be done in this initialisation phase. (Lines 29-35)
        \item The benchmark must start a timer, either using timers provided by
            the PMM shared library, libpmm, or using his own methods (Lines
            38-41)
        \item The benchmark must execute the routine directly after timing is
            initiated (Line 44)
        \item the benchmark must stop timing directly after the routine has
            finished (Line 47)
        \item the benchmark must print on a single line, to \verb+stdout+, the
            seconds and microseconds, separated by a single space, elapsed
            during the routine execution. This can be done using a function
            provided by libpmm or the users own method. (Line 50)
        \item the benchmark must print on a new line, the volume
            of computations made by the routine (typically the number of
            floating point operations carried out). Long long integers are
            supported.  (Lines 26,38,50)
        \item on successful completion of the above operations, the benchmark
            should terminate and return successful exit status, PMM expects
            this to be equivalent to \verb+EXIT_SUCCESS+ as defined by the C
            standard. (Line 59)
    \end{itemize}

    Listing \ref{square_mxm_code} shows an example benchmark for a square
    matrix multiplication routine provided by GSL. The inline comments refer to
    each of the points made above

    \begin{lstlisting}[style=c_code,caption=Square Matrix Multiplication Benchmark,label=square_mxm_code]
#include <stdlib.h>
#include <stdio.h>
#include "pmm_util.h"
#include <gsl/gsl_blas.h>

#define NARGS 1

int main(int argc, char **argv) {

	/* declare variables */
	gsl_matrix *A, *B, *C;
	double arg;
	size_t n;
	long long int c;

	/* parse arguments */
	if(argc != NARGS+1) {
		return PMM_EXIT_ARGFAIL;
	}
	if(sscanf(argv[1], "%lf", &arg) == 0) {
		return PMM_EXIT_ARGPARSEFAIL;
	}
	n = (size_t)arg;

	/* calculate complexity */
	c = 2*n*n*(long long int)n;

	/* initialise data */
	A = gsl_matrix_alloc(n, n);
	B = gsl_matrix_alloc(n, n);
	C = gsl_matrix_alloc(n, n);

	gsl_matrix_set_all(A, 2.5);
	gsl_matrix_set_all(B, 4.9);
	gsl_matrix_set_zero(C);

	/* initialise timer */
	pmm_timer_init(c);

	/* start timer */
	pmm_timer_start();

	/* execute routine */
	gsl_blas_dgemm(CblasNoTrans, CblasNoTrans, 1.0, A, B, 0.0, C);

	/* stop timer */
	pmm_timer_stop();

	/* get timing result */
	pmm_timer_result();

	/* destroy timer */
	pmm_timer_destroy();

	gsl_matrix_free(A);
	gsl_matrix_free(B);
	gsl_matrix_free(C);

	return PMM_EXIT_SUCCESS;
}
    \end{lstlisting}

    \section{Configuring the Benchmark in PMM}

    For now, use the example configurations as a template. Small parameter
    strides should be avoided as they make naive/optimised construction very
    slow.

    \include{fdl-1.3}

\end{document}

