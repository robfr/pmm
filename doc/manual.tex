\documentclass[a4paper,12pt]{report}

\usepackage[english]{babel} %language selection
\selectlanguage{english}

%\usepackage{html,makeidx}
\usepackage{pslatex}
\usepackage{boxedminipage}
\usepackage{listings}

\usepackage{titlesec}
\titleformat{\chapter}[hang]{\bf\huge}{\thechapter}{2pc}{}

\pagenumbering{arabic}

\usepackage{hyperref}
\hypersetup{colorlinks, 
           citecolor=black,
           filecolor=black,
           linkcolor=black,
           urlcolor=black,
           bookmarksopen=true}
           
\usepackage{html}   %  *always* load this for LaTeX2HTML
\begin{htmlonly}
  \usepackage{verbatim}
  \providecommand{\lstinputlisting}[2][]{\verbatiminput{#2}}
\end{htmlonly}


%\hfuzz = .6pt % avoid black boxes

\newcommand{\devvar}{The default value is suitable and this variable is made configurable mostly for developmental purposes}

% style for XML Config listings
\lstdefinestyle{xmlconfig}{language=XML,basicstyle=\small\ttfamily,numbers=left,frame=lines,breaklines=true}

% style for C Code listings
\lstdefinestyle{c_code}{language=C,basicstyle=\small\ttfamily,numbers=left,frame=lines,breaklines=true}

\begin{document}

    \pagestyle{empty}
    \vspace*{15cm}
    \begin{flushright}
        \rule[-1ex]{11cm}{3pt}\\
        \huge Performance Model Manager\\
        \huge \emph{User Manual}\\
    \end{flushright}

    \newpage


    \noindent

    \begin{boxedminipage}[b]{\textwidth}
    Performance Model Manager User Manual,
    Copyright \copyright{} 2010 Robert Higgins.
    Permission is granted to copy, distribute and/or modify this document
    under the \\ terms of the GNU Free Documentation License, Version 1.3
    or any later version published by the Free Software Foundation;
    with no Invariant Sections, no Front-Cover Texts, and no Back-Cover Texts.
    A copy of the license is included in the section entitled ``GNU
    Free Documentation License''.
    \end{boxedminipage}

    \newpage

    \tableofcontents


    \chapter{Introduction}

    Performance Model Manager (PMM) is an open source GNU Public Licenced tool
    for experimentation in the use of Functional Performance Models (FPMs). An
    FPM describes the speed of a computational routine in terms of the
    routine's input parameters. It focuses on addressing issues surrounding the
    construction, maintenance and use of FPMs. To this end, it has three main
    features:
    \begin{itemize}
        \item It implements the Geometric Bisection Building Procedure for
            multi-parameter FPMs, optimizing the construction of a problem's
            performance model. 
        \item It permits the construction of models for a large number of
            problems by implementing a flexible benchmarking scheduler.
        \item It provides access to the models in a variety of ways, so that
            they may be visualised or used to make scheduling decisions.
    \end{itemize}
    Construction of the FPM of some general computational procedure is supported
    by requiring that the user provides a benchmarking executable which behaves
    according to a simple protocol.
    
    An example implementation of such a 'benchmark binary' is given in Chapter
    \ref{chap:writingbench} along with details of its configuration within PMM.
    Further examples are also included in the source distribution.

    Models can be constructed on demand or in the background by the \verb+pmmd+
    daemon. Construction of multiple models can be scheduled according to a
    variety of policies. They will be constructed in turn according to their
    priority and scheduling criteria.

    Access to models will be made available via an API in a future release of
    PMM, at present only viewing of models is possible, via a plotting program:
    \verb+pmm_view+.

    This manual continues to Chapter \ref{install_chap} where compilation and 
    installation is described. Then the PMM configuration file is described in
    Chapter \ref{config_chap}. Chapter \ref{bench_chap} provides notes on how
    to write a benchmark binary and configure it as a 'routine' to be modelled
    within
    PMM.

    \chapter{Installation}
    \label{install_chap}

    \section{Requirements}
    PMM is developed for the Linux platform but may also compile on other POSIX
    operating systems. The following softwares are required to install PMM:
    \begin{itemize}
        \item Gnuplot
        \item GNU Make
        \item GCC compiler suite (tested with 4.x series only)
        \item libxml2 2.6.0 or greater
        \item GNU Scientific Library
    \end{itemize}
    The following are optional but enable certain features:
    \begin{itemize}
        \item Octave (2.9.14 or greater) is required for multi-parameter model construction
        \item PAPI (4.0.0 tested) is required for higher resolution timing and automatic complexity calculation
        \item GotoBLAS2 is required for for further example problems
    \end{itemize}

    \section{Compiling \& Installing}
    Installation of PMM uses a hierarchy of directories under a certain prefix,
    by default \verb+/usr+. If this is not desirable the build should be
    configured with the \verb+--prefix=<dir>+ option. A typical installation
    follows:

    \begin{verbatim}
        $ tar -xzf pmm-0.0.1.tar.gz
        $ cd pmm-0.0.1
        $ ./configure --prefix=$PWD/install
        $ make && make install
    \end{verbatim}

    \noindent Configuration options to note:

    \begin{itemize}
        \item \verb+--enable-debug+ enable debugging messages and flags
        \item \verb+--disable-octave+ disable use of octave and multi-parameter
            model support
        \item \verb+--disable-benchgslblas+ disable compilation and
            installation of demonstration GSL problem benchmarks
        \item \verb+--enable-benchgotoblas2+ enable compilation and
            installation of demonstration GotoBLAS2 problems
        \item \verb+--with-papi[=path]+ enable use of PAPI with optional
            specification of PAPI installation path
    \end{itemize}

    \noindent Further options can be viewed by running \verb+./configure --help+
    .

    After installation, the PMM daemon is started by executing the
    \verb+pmmd+ binary and the PMM viewer program is run via the \verb+pmm_view+
    binary.


    \chapter{Configuration}
    \label{config_chap}
    PMM is distributed with a default configuration which will be installed under:
    \begin{verbatim} <prefix>/etc/pmmd.conf[.sample] \end{verbatim}

    \noindent This can serve as a template for a user's own configuration and
    contains sane values for all options. If example problem benchmarks are
    built, the sample configuration will also describe those problems.
    
    The configuration file has an heirarchical XML structure Configuration is
    described between \verb+<config>+ root element tags.  Under this, the load
    monitor facility is described by a \verb+<load_monitor>+ element, and each
    routine for which a model is to be built, is described by a 
    \verb+<routine>+ element.

    In the following sections, each element in the configuration file is
    described. If an element has a default value, it need not be explicitly set
    in the configuration file, on the other hand, some options must be set.
    This information, along with the type of the expected element value
    (string, integer, etc.) and what exactly the element describes is detailed
    below.

    \section{General Configuration}
    The following elements (which can be seen in context in Listing
    \ref{basic_config_example}) define some general application configurable
    options and come directly under the \verb+<config>+ tags:
    \begin{itemize}
        \item \verb+<main_sleep_period>+ (\emph{integer, default:1}) The
            benchmark scheduler checks the system state ever $n$ seconds and
            this period may be configured here. \devvar.
        \item \verb+<model_write_time_threshold>+ (\emph{integer, default:60})
            When benchmarking problems of very small size, which execute very
            quickly, the manager may become overloaded by writing the model to
            disk after each execution. This option allows us to configure how
            often the model will be saved to disk, i.e. after a total of $n$
            seconds has been spent benchmarking a particular model it will be
            written to disk.  \devvar.
        \item \verb+<model_write_execs_threshold>+ (\emph{integer, default:10})
            This option serves the same purpose as the previous one, except
            that it specifies the number of benchmark executions that must
            occur before it is written to disk.  It may take hundreds of small
            benchmarks exceed the time threshold (above), so this second
            threshold allows us to write based on execution frequency as well.
            \devvar.
    \end{itemize}

    \begin{lstlisting}[style=xmlconfig,caption=Basic Configuration,float=h,label=basic_config_example]
<?xml version="1.0"?>
<config>
   <main_sleep_period>1</main_sleep_period>
   <model_write_time_threshold>60
      </model_write_time_threshold>
   <model_write_execs_threshold>20
      </model_write_execs_threshold>

   <load_monitor>
      <load_path>/usr/var/pmm/loadhistory</load_path>
      <write_period>60</write_period>
      <history_size>60</history_size>
   </load_monitor>

   ....
</config>
    \end{lstlisting}

    \section{Load Monitor Configuration}

    The load monitoring facility is is described by a \verb+<load_monitor>+
    element, this has the following children:

    \begin{itemize}
        \item \verb+<load_path>+ (\emph{string, required}) path to a file where
            load observations are recorded
        \item \verb+<write_period>+ (\emph{integer, default:360}) frequency
            with which to save the load file to disk (in seconds)
        \item \verb+<history_size>+ (\emph{integer, default:60}) number of load
            observations to store \end{itemize}


    \section{Routine Configuration}

    \noindent Each routine is described by a \verb+<routine>+ element.
    Routines have detailed descriptions of the parameters to be passed to them
    and the construction method that should be used to build their models. An
    example routine configuration can been seen in Listing
    \ref{routine_config_example}, it describes a 2-parameter routine. First,
    the general options are set using the following child elements:
    \begin{itemize}
        \item \verb+<name>+ (\emph{string, required}) The routine name.
        \item \verb+<exe_path>+ (\emph{string, required}) The path to the
            benchmarking executable
        \item \verb+<model_path>+ (\emph{string, required}) The path to the
            file where the performance model of the routine will be saved.
        \item \verb+<priority>+ (\emph{integer, default:0}) The construction
            priority this routine has (logically, the higher the value, the
            higher the priority)
    \end{itemize}

    \noindent The parameters of a routine are described by a 
    \verb+<parameters>+ elmenet. These are the parameters that will be passed to
    the benchmark binary which ultimately executes the routine which is being
    modeled. The parameters passed to the benchmark are those that influnce the
    volume of computations or the speed at which the computations are carried
    outt. This is further described in Chapter \ref{bench_chap}. The first
    child of the \verb+<parameters>+ element must be the number of parameter
    descriptions which will follow:
    \begin{itemize}
        \item \verb+<n_p>+ (\emph{integer, required}) number of parameters
            which the benchmark accepts
    \end{itemize}
    Following that, each parameter is described by a \verb+<param>+ element.
    The \verb+<param>+ element has a number of child elements which are:
    \begin{itemize}
        \item \verb+<order>+ (\emph{integer, required}) The order
            in which this parameter should be passed to the benchmark binary.
        \item \verb+<name>+ (\emph{string, required}) The name of this
            parameter.
        \item \verb+<min>+ (\emph{integer, required}) The
            minimum value this parameter may have. If modelling the performance
            of the processor while operating in cache only is \emph{not}
            important, this should be set so the overall problem size is large
            enough to occupy main memory.
        \item \verb+<max>+ (\emph{integer, required}) The maximum value this
            parameter may have. This should be large enough to induce
            significant paging.
        \item \verb+<stride>+ (\emph{integer, default:1}) The stride with which
            this parameter should be incremented. Stride influences the
            climbing phase of optimised construction (where successive
            benchmarks are incremented in size by this value) as well as naive
            construction (where all points on the stride between min and max
            are benchmarked). A reasonable value for stride would be, for
            example, 1/100th of the range between max and min. If stride is too
            low, excessive time may be spent building a model.
        \item \verb+<offset>+ (\emph{integer, default:0}) Offset for this
            parameter. If required, you can specify that a parameter value must
            always be a certain offset from zero.
        \item \verb+<fuzzy_max>+ (\emph{boolean, default:false}) This specifies
            that the maximum parameter size defined is not a true max and speed
            at this maximum should be measured.
            
            In normal circumstances the FPM is constructed across
            a complete range of problem sizes, from small to so large that
            speed is effectively zero. The maximum parameter value will be so
            large that it induces heavy paging. Speed at this maximum is not
            measured, but assumed to be zero. If this is \emph{not} the case,
            and the maximum parameter size will not induce heavy paging,
            \verb+<fuzzy_max>+ must be set to \emph{true} for the GBBP algorithm
            to complete successfully.
    \end{itemize}


    \noindent Directives for the construction method must be described by a
    \verb+<construction>+ element. It has the following child elements:
    \begin{itemize}
        \item \verb+<method>+ (\emph{string, default:gbbp}) The construction
            method, this element may have the following values:
        \begin{itemize}
            \item \emph{gbbp} - the Geometric Bisection Building Procedure will
                be used to select benchmark points, minimising the number of
                points required to accurately estimate the model
            \item \emph{naive} - all possible points between the parameter
                ranges will be benchmarked.
            \item \emph{rand} - points between the parameter ranges will be
                selected at random
        \end{itemize}
        \item \verb+<min_sample_num>+ (\emph{integer, default:1}) Specify the
            minimum number of benchmarks to be taken at a single point in the
            model. Once this is met, the point will be considered as measured
            and the construction method will proceed to the next point of its
            choosing.
        \item \verb+<min_sample_time>+ (\emph{integer, default:0}) Specify the
            minimum number of seconds that should be spent in the benchmarking
            of a single point before it is considered as measured. I.e. if set
            to 60 seconds, a benchmark taking 20 seconds will be measured 3
            times.
    \end{itemize}

    Finally, priority and scheduling policy may be specified. When multiple
    routines are configured in PMM priorities allow the user to specify which
    models will be built first. Scheduling policies allow the user limit the
    execution of benchmarks to certain time periods or certain system
    conditions.

    \begin{itemize}
        \item \verb+<priority>+ (\emph{integer, default:0}) Priority of
            construction for the routine. Higher priority routines will have
            their models constructed before lower ones
        \item \verb+<condition>+ (\emph{string, default:now}) Condition under
            which benchmarking of a routine is permitted. Note: Once started, a
            benchmark will not be interrupted, even if the conditions that
            permitted its execution have changed to ones which would otherwise
            prevent execution.
            \begin{itemize}
                \item \emph{now} - construction is permitted at all times
                \item \emph{idle} - construction is only permitted when the
                    observed 5 minute load average is less than 0.10 (note: the
                    act of benchmarking will influence the load average of the
                    system. After the benchmark is complete, PMM will probably
                    have to wait ~5 minutes before the next execution can
                    occur)
                \item \emph{nousers} - construction is only permitted when no
                    users are logged into the system. Logged in users would be
                    those reported by utilities such as \verb+w+, \verb+who+,
                    \verb+users+ and so on.
            \end{itemize}
    \end{itemize}


    \begin{lstlisting}[style=xmlconfig,caption=Routine Configuration Example,label=routine_config_example]
<routine>
   <name>dgemm2</name>
   <exe_path>/usr/local/lib/pmm/dgemm2</exe_path>
   <model_path>/usr/local/var/pmm/dgemm2_model</model_path>
   <parameters>
      <n_p>2</n_p>
      <param>
         <order>0</order>
         <name>m</name>
         <min>32</min>
         <max>4096</max>
         <stride>32</stride>
         <offset>0</offset>
      </param>
      <param>
         <order>1</order>
         <name>n</name>
         <min>32</min>
         <max>4096</max>
         <stride>32</stride>
         <offset>0</offset>
      </param>
   </parameters>
   <construction>
      <method>gbbp</method>
      <min_sample_num>5</min_sample_num>
      <min_sample_time>120</min_sample_time>
   </construction>
   <condition>now</condition>
   <priority>30</priority>
</routine>
    \end{lstlisting}

    \chapter{Building the FPM of a Computation}
    \label{bench_chap}
    This chapter outlines what a user must do to have PMM build the FPM of some
    computation. The computation may be a library subroutine, a code
    fragment or an entire process. Throughout this document this computation
    will be referred to as a \emph{routine}. The users routine must be wrapped
    in a benchmarking binary or script which should behave in a specific
    manner:

    \begin{itemize}
        \item It must accept arguments from the command line which define
            the volume of computations it must carry out.
        \item It must execute and time the computation that is to be modeled.
            Execution may be via a script or compiled binary, written in any
            language, and the details of how it perform or times the
            computation do not concern the PMM tool. If the benchmark is written
            in C/C++, PMM provides some utilities to aid this in a shared
            library, \verb+libpmm+.
        \item It must output timing and volume of computations (complexity) in
            a standard manner. \verb+libpmm+ also supports this.
    \end{itemize}

    Implementation of this benchmark is a task left to the user. The following
    sections describe how to choose input parameters, write the benchmark and
    configure PMM to build a FPM for the routine.

    \section{Choosing Parameters of a Routine}

    The first step a user must take is to identify the parameters of the
    routine which effect the volume of computations it must carry out.
    Typically, the volume of computations would be floating point operation
    count, however PMM is agnostic to the type of computations the routine
    carries out, and the volume may be expressed as the user wishes.
    The performance model that we build will be expressed in terms of these
    parameters. Throughout this chapter we will refer to an example of a square
    matrix multiplication. In this scenario, there is only one parameter that
    effects the volume of computations, $N$, the length of a matrix side in the
    multiplication.

    For a more general case, were two matrices of sizes $N \times K$ and $K
    \times M$ are multiplied and the result stored in an third matrix of size
    $N \times M$, then the volume of computations would depend on three
    parameters, $N$, $M$ and $K$.

    A general purpose matrix multiplication routine usually has other
    associated parameters defining transpositions of the input data and other
    coefficients. These however do not contribute significantly to the
    computational complexity of the routine and they should not be considered
    as parameters of the model in the PMM framework. It is important to note
    than building an FPM which is in terms of more than one parameter is very
    intensive as the number of points required to accurately approximate scales
    exponentially with the number of parameters the model is in terms of. Any
    parameters of a routine that can be excluded from the functional
    performance model should be.

    \section{Writing a Benchmark for PMM} \label{chap:writingbench}

    For PMM to build the performance model of a routine, it must be able to
    execute benchmarks of that routine for various problem sizes. As previously
    stated, the problem's size is determined by the parameters which effect the
    computational complexity of the routine, and the performance model is a
    function of these parameters.
    
    PMM must be provided with an executable which carries out a benchmark with
    given input parameters.  The user must write this executable so that it
    behaves in a specified way. PMM is distributed with the source of a number
    of example benchmarks, here we will list one and reference it as the
    required behaviours are described below. Listing \ref{square_mxm_code}
    shows an example benchmark for a square matrix multiplication routine. The
    multiplication is provided by the Gnu Scientific Library. Note in a square
    matrix multiplication, the volume of computations is determined by the size
    of one side of matrices to be multiplied.

    The benchmark code behaves in the following manner:

    \begin{itemize}
        \item The executable must accept a number of parameters on the command
            line. These parameters will also be described in the configuration
            entry for the routine. As of version 0.0.1 parameters can only have
            integer types. (Lines 17-23)
        \item Based on the parameters passed on the command line, the benchmark
            must initialise memory and data structures that are to be passed to
            the routine. If the computation that is to be modelled is just a
            simple code fragment, no allocation of memory that occurs within
            the code fragment should be done in this initialisation phase. (Lines 29-35)
        \item The benchmark must start a timer, either using timers provided by
            the PMM shared library, libpmm, or using his own methods (Lines
            38-41)
        \item The benchmark must execute the routine directly after timing is
            initiated (Line 44)
        \item the benchmark must stop timing directly after the routine has
            finished (Line 47)
        \item the benchmark must print on a single line, to \verb+stdout+, the
            seconds and microseconds, separated by a single space, elapsed
            during the routine execution. This can be done using a function
            provided by libpmm or the users own method. (Line 50)
        \item the benchmark must print on a new line, the volume
            of computations made by the routine (typically the number of
            floating point operations carried out). Long long integers are
            supported.  (Lines 26,38,50)
        \item on successful completion of the above operations, the benchmark
            should terminate and return successful exit status, PMM expects
            this to be equivalent to \verb+EXIT_SUCCESS+ as defined by the C
            standard. (Line 59)
    \end{itemize}

    Listing \ref{square_mxm_code} shows an example benchmark for a square
    matrix multiplication routine provided by GSL. The inline comments refer to
    each of the points made above

    \begin{lstlisting}[style=c_code,caption=Square Matrix Multiplication Benchmark,label=square_mxm_code]
#include <stdlib.h>
#include <stdio.h>
#include "pmm_util.h"
#include <gsl/gsl_blas.h>

#define NARGS 1

int main(int argc, char **argv) {

	/* declare variables */
	gsl_matrix *A, *B, *C;
	double arg;
	size_t n;
	long long int c;

	/* parse arguments */
	if(argc != NARGS+1) {
		return PMM_EXIT_ARGFAIL;
	}
	if(sscanf(argv[1], "%lf", &arg) == 0) {
		return PMM_EXIT_ARGPARSEFAIL;
	}
	n = (size_t)arg;

	/* calculate complexity */
	c = 2*n*n*(long long int)n;

	/* initialise data */
	A = gsl_matrix_alloc(n, n);
	B = gsl_matrix_alloc(n, n);
	C = gsl_matrix_alloc(n, n);

	gsl_matrix_set_all(A, 2.5);
	gsl_matrix_set_all(B, 4.9);
	gsl_matrix_set_zero(C);

	/* initialise timer */
	pmm_timer_init(c);

	/* start timer */
	pmm_timer_start();

	/* execute routine */
	gsl_blas_dgemm(CblasNoTrans, CblasNoTrans, 1.0, A, B, 0.0, C);

	/* stop timer */
	pmm_timer_stop();

	/* get timing result */
	pmm_timer_result();

	/* destroy timer */
	pmm_timer_destroy();

	gsl_matrix_free(A);
	gsl_matrix_free(B);
	gsl_matrix_free(C);

	return PMM_EXIT_SUCCESS;
}
    \end{lstlisting}

    \section{Configuring the Benchmark in PMM}

    For now, use the example configurations as a template. Small parameter
    strides should be avoided as they make naive/optimised construction very
    slow.

    % This is set up to run with pdflatex.
%---------The file header---------------------------------------------
%\documentclass[a4paper,12pt]{book}
%
%\usepackage[english]{babel} %language selection
%\selectlanguage{english}
%
%\pagenumbering{arabic}
%
%\usepackage{hyperref}
%\hypersetup{colorlinks, 
%           citecolor=black,
%           filecolor=black,
%           linkcolor=black,
%           urlcolor=black,
%           bookmarksopen=true,
%           pdftex}
%
%\hfuzz = .6pt % avoid black boxes
%           
%\begin{document}
%---------------------------------------------------------------------
\chapter*{\rlap{GNU Free Documentation License}}
\phantomsection  % so hyperref creates bookmarks
\addcontentsline{toc}{chapter}{GNU Free Documentation License}
%\label{label_fdl}

 \begin{center}

       Version 1.3, 3 November 2008


 Copyright \copyright{} 2000, 2001, 2002, 2007, 2008  Free Software Foundation, Inc.
 
 \bigskip
 
     $<$http://fsf.org/$>$
  
 \bigskip
 
 Everyone is permitted to copy and distribute verbatim copies
 of this license document, but changing it is not allowed.
\end{center}


\begin{center}
{\bf\large Preamble}
\end{center}

The purpose of this License is to make a manual, textbook, or other
functional and useful document ``free'' in the sense of freedom: to
assure everyone the effective freedom to copy and redistribute it,
with or without modifying it, either commercially or noncommercially.
Secondarily, this License preserves for the author and publisher a way
to get credit for their work, while not being considered responsible
for modifications made by others.

This License is a kind of ``copyleft'', which means that derivative
works of the document must themselves be free in the same sense.  It
complements the GNU General Public License, which is a copyleft
license designed for free software.

We have designed this License in order to use it for manuals for free
software, because free software needs free documentation: a free
program should come with manuals providing the same freedoms that the
software does.  But this License is not limited to software manuals;
it can be used for any textual work, regardless of subject matter or
whether it is published as a printed book.  We recommend this License
principally for works whose purpose is instruction or reference.


\begin{center}
{\Large\bf 1. APPLICABILITY AND DEFINITIONS\par}
\phantomsection
\addcontentsline{toc}{section}{1. APPLICABILITY AND DEFINITIONS}
\end{center}

This License applies to any manual or other work, in any medium, that
contains a notice placed by the copyright holder saying it can be
distributed under the terms of this License.  Such a notice grants a
world-wide, royalty-free license, unlimited in duration, to use that
work under the conditions stated herein.  The ``\textbf{Document}'', below,
refers to any such manual or work.  Any member of the public is a
licensee, and is addressed as ``\textbf{you}''.  You accept the license if you
copy, modify or distribute the work in a way requiring permission
under copyright law.

A ``\textbf{Modified Version}'' of the Document means any work containing the
Document or a portion of it, either copied verbatim, or with
modifications and/or translated into another language.

A ``\textbf{Secondary Section}'' is a named appendix or a front-matter section of
the Document that deals exclusively with the relationship of the
publishers or authors of the Document to the Document's overall subject
(or to related matters) and contains nothing that could fall directly
within that overall subject.  (Thus, if the Document is in part a
textbook of mathematics, a Secondary Section may not explain any
mathematics.)  The relationship could be a matter of historical
connection with the subject or with related matters, or of legal,
commercial, philosophical, ethical or political position regarding
them.

The ``\textbf{Invariant Sections}'' are certain Secondary Sections whose titles
are designated, as being those of Invariant Sections, in the notice
that says that the Document is released under this License.  If a
section does not fit the above definition of Secondary then it is not
allowed to be designated as Invariant.  The Document may contain zero
Invariant Sections.  If the Document does not identify any Invariant
Sections then there are none.

The ``\textbf{Cover Texts}'' are certain short passages of text that are listed,
as Front-Cover Texts or Back-Cover Texts, in the notice that says that
the Document is released under this License.  A Front-Cover Text may
be at most 5 words, and a Back-Cover Text may be at most 25 words.

A ``\textbf{Transparent}'' copy of the Document means a machine-readable copy,
represented in a format whose specification is available to the
general public, that is suitable for revising the document
straightforwardly with generic text editors or (for images composed of
pixels) generic paint programs or (for drawings) some widely available
drawing editor, and that is suitable for input to text formatters or
for automatic translation to a variety of formats suitable for input
to text formatters.  A copy made in an otherwise Transparent file
format whose markup, or absence of markup, has been arranged to thwart
or discourage subsequent modification by readers is not Transparent.
An image format is not Transparent if used for any substantial amount
of text.  A copy that is not ``Transparent'' is called ``\textbf{Opaque}''.

Examples of suitable formats for Transparent copies include plain
ASCII without markup, Texinfo input format, LaTeX input format, SGML
or XML using a publicly available DTD, and standard-conforming simple
HTML, PostScript or PDF designed for human modification.  Examples of
transparent image formats include PNG, XCF and JPG.  Opaque formats
include proprietary formats that can be read and edited only by
proprietary word processors, SGML or XML for which the DTD and/or
processing tools are not generally available, and the
machine-generated HTML, PostScript or PDF produced by some word
processors for output purposes only.

The ``\textbf{Title Page}'' means, for a printed book, the title page itself,
plus such following pages as are needed to hold, legibly, the material
this License requires to appear in the title page.  For works in
formats which do not have any title page as such, ``Title Page'' means
the text near the most prominent appearance of the work's title,
preceding the beginning of the body of the text.

The ``\textbf{publisher}'' means any person or entity that distributes
copies of the Document to the public.

A section ``\textbf{Entitled XYZ}'' means a named subunit of the Document whose
title either is precisely XYZ or contains XYZ in parentheses following
text that translates XYZ in another language.  (Here XYZ stands for a
specific section name mentioned below, such as ``\textbf{Acknowledgements}'',
``\textbf{Dedications}'', ``\textbf{Endorsements}'', or ``\textbf{History}''.)  
To ``\textbf{Preserve the Title}''
of such a section when you modify the Document means that it remains a
section ``Entitled XYZ'' according to this definition.

The Document may include Warranty Disclaimers next to the notice which
states that this License applies to the Document.  These Warranty
Disclaimers are considered to be included by reference in this
License, but only as regards disclaiming warranties: any other
implication that these Warranty Disclaimers may have is void and has
no effect on the meaning of this License.


\begin{center}
{\Large\bf 2. VERBATIM COPYING\par}
\phantomsection
\addcontentsline{toc}{section}{2. VERBATIM COPYING}
\end{center}

You may copy and distribute the Document in any medium, either
commercially or noncommercially, provided that this License, the
copyright notices, and the license notice saying this License applies
to the Document are reproduced in all copies, and that you add no other
conditions whatsoever to those of this License.  You may not use
technical measures to obstruct or control the reading or further
copying of the copies you make or distribute.  However, you may accept
compensation in exchange for copies.  If you distribute a large enough
number of copies you must also follow the conditions in section~3.

You may also lend copies, under the same conditions stated above, and
you may publicly display copies.


\begin{center}
{\Large\bf 3. COPYING IN QUANTITY\par}
\phantomsection
\addcontentsline{toc}{section}{3. COPYING IN QUANTITY}
\end{center}


If you publish printed copies (or copies in media that commonly have
printed covers) of the Document, numbering more than 100, and the
Document's license notice requires Cover Texts, you must enclose the
copies in covers that carry, clearly and legibly, all these Cover
Texts: Front-Cover Texts on the front cover, and Back-Cover Texts on
the back cover.  Both covers must also clearly and legibly identify
you as the publisher of these copies.  The front cover must present
the full title with all words of the title equally prominent and
visible.  You may add other material on the covers in addition.
Copying with changes limited to the covers, as long as they preserve
the title of the Document and satisfy these conditions, can be treated
as verbatim copying in other respects.

If the required texts for either cover are too voluminous to fit
legibly, you should put the first ones listed (as many as fit
reasonably) on the actual cover, and continue the rest onto adjacent
pages.

If you publish or distribute Opaque copies of the Document numbering
more than 100, you must either include a machine-readable Transparent
copy along with each Opaque copy, or state in or with each Opaque copy
a computer-network location from which the general network-using
public has access to download using public-standard network protocols
a complete Transparent copy of the Document, free of added material.
If you use the latter option, you must take reasonably prudent steps,
when you begin distribution of Opaque copies in quantity, to ensure
that this Transparent copy will remain thus accessible at the stated
location until at least one year after the last time you distribute an
Opaque copy (directly or through your agents or retailers) of that
edition to the public.

It is requested, but not required, that you contact the authors of the
Document well before redistributing any large number of copies, to give
them a chance to provide you with an updated version of the Document.


\begin{center}
{\Large\bf 4. MODIFICATIONS\par}
\phantomsection
\addcontentsline{toc}{section}{4. MODIFICATIONS}
\end{center}

You may copy and distribute a Modified Version of the Document under
the conditions of sections 2 and 3 above, provided that you release
the Modified Version under precisely this License, with the Modified
Version filling the role of the Document, thus licensing distribution
and modification of the Modified Version to whoever possesses a copy
of it.  In addition, you must do these things in the Modified Version:

\begin{itemize}
\item[A.] 
   Use in the Title Page (and on the covers, if any) a title distinct
   from that of the Document, and from those of previous versions
   (which should, if there were any, be listed in the History section
   of the Document).  You may use the same title as a previous version
   if the original publisher of that version gives permission.
   
\item[B.]
   List on the Title Page, as authors, one or more persons or entities
   responsible for authorship of the modifications in the Modified
   Version, together with at least five of the principal authors of the
   Document (all of its principal authors, if it has fewer than five),
   unless they release you from this requirement.
   
\item[C.]
   State on the Title page the name of the publisher of the
   Modified Version, as the publisher.
   
\item[D.]
   Preserve all the copyright notices of the Document.
   
\item[E.]
   Add an appropriate copyright notice for your modifications
   adjacent to the other copyright notices.
   
\item[F.]
   Include, immediately after the copyright notices, a license notice
   giving the public permission to use the Modified Version under the
   terms of this License, in the form shown in the Addendum below.
   
\item[G.]
   Preserve in that license notice the full lists of Invariant Sections
   and required Cover Texts given in the Document's license notice.
   
\item[H.]
   Include an unaltered copy of this License.
   
\item[I.]
   Preserve the section Entitled ``History'', Preserve its Title, and add
   to it an item stating at least the title, year, new authors, and
   publisher of the Modified Version as given on the Title Page.  If
   there is no section Entitled ``History'' in the Document, create one
   stating the title, year, authors, and publisher of the Document as
   given on its Title Page, then add an item describing the Modified
   Version as stated in the previous sentence.
   
\item[J.]
   Preserve the network location, if any, given in the Document for
   public access to a Transparent copy of the Document, and likewise
   the network locations given in the Document for previous versions
   it was based on.  These may be placed in the ``History'' section.
   You may omit a network location for a work that was published at
   least four years before the Document itself, or if the original
   publisher of the version it refers to gives permission.
   
\item[K.]
   For any section Entitled ``Acknowledgements'' or ``Dedications'',
   Preserve the Title of the section, and preserve in the section all
   the substance and tone of each of the contributor acknowledgements
   and/or dedications given therein.
   
\item[L.]
   Preserve all the Invariant Sections of the Document,
   unaltered in their text and in their titles.  Section numbers
   or the equivalent are not considered part of the section titles.
   
\item[M.]
   Delete any section Entitled ``Endorsements''.  Such a section
   may not be included in the Modified Version.
   
\item[N.]
   Do not retitle any existing section to be Entitled ``Endorsements''
   or to conflict in title with any Invariant Section.
   
\item[O.]
   Preserve any Warranty Disclaimers.
\end{itemize}

If the Modified Version includes new front-matter sections or
appendices that qualify as Secondary Sections and contain no material
copied from the Document, you may at your option designate some or all
of these sections as invariant.  To do this, add their titles to the
list of Invariant Sections in the Modified Version's license notice.
These titles must be distinct from any other section titles.

You may add a section Entitled ``Endorsements'', provided it contains
nothing but endorsements of your Modified Version by various
parties---for example, statements of peer review or that the text has
been approved by an organization as the authoritative definition of a
standard.

You may add a passage of up to five words as a Front-Cover Text, and a
passage of up to 25 words as a Back-Cover Text, to the end of the list
of Cover Texts in the Modified Version.  Only one passage of
Front-Cover Text and one of Back-Cover Text may be added by (or
through arrangements made by) any one entity.  If the Document already
includes a cover text for the same cover, previously added by you or
by arrangement made by the same entity you are acting on behalf of,
you may not add another; but you may replace the old one, on explicit
permission from the previous publisher that added the old one.

The author(s) and publisher(s) of the Document do not by this License
give permission to use their names for publicity for or to assert or
imply endorsement of any Modified Version.


\begin{center}
{\Large\bf 5. COMBINING DOCUMENTS\par}
\phantomsection
\addcontentsline{toc}{section}{5. COMBINING DOCUMENTS}
\end{center}


You may combine the Document with other documents released under this
License, under the terms defined in section~4 above for modified
versions, provided that you include in the combination all of the
Invariant Sections of all of the original documents, unmodified, and
list them all as Invariant Sections of your combined work in its
license notice, and that you preserve all their Warranty Disclaimers.

The combined work need only contain one copy of this License, and
multiple identical Invariant Sections may be replaced with a single
copy.  If there are multiple Invariant Sections with the same name but
different contents, make the title of each such section unique by
adding at the end of it, in parentheses, the name of the original
author or publisher of that section if known, or else a unique number.
Make the same adjustment to the section titles in the list of
Invariant Sections in the license notice of the combined work.

In the combination, you must combine any sections Entitled ``History''
in the various original documents, forming one section Entitled
``History''; likewise combine any sections Entitled ``Acknowledgements'',
and any sections Entitled ``Dedications''.  You must delete all sections
Entitled ``Endorsements''.

\begin{center}
{\Large\bf 6. COLLECTIONS OF DOCUMENTS\par}
\phantomsection
\addcontentsline{toc}{section}{6. COLLECTIONS OF DOCUMENTS}
\end{center}

You may make a collection consisting of the Document and other documents
released under this License, and replace the individual copies of this
License in the various documents with a single copy that is included in
the collection, provided that you follow the rules of this License for
verbatim copying of each of the documents in all other respects.

You may extract a single document from such a collection, and distribute
it individually under this License, provided you insert a copy of this
License into the extracted document, and follow this License in all
other respects regarding verbatim copying of that document.


\begin{center}
{\Large\bf 7. AGGREGATION WITH INDEPENDENT WORKS\par}
\phantomsection
\addcontentsline{toc}{section}{7. AGGREGATION WITH INDEPENDENT WORKS}
\end{center}


A compilation of the Document or its derivatives with other separate
and independent documents or works, in or on a volume of a storage or
distribution medium, is called an ``aggregate'' if the copyright
resulting from the compilation is not used to limit the legal rights
of the compilation's users beyond what the individual works permit.
When the Document is included in an aggregate, this License does not
apply to the other works in the aggregate which are not themselves
derivative works of the Document.

If the Cover Text requirement of section~3 is applicable to these
copies of the Document, then if the Document is less than one half of
the entire aggregate, the Document's Cover Texts may be placed on
covers that bracket the Document within the aggregate, or the
electronic equivalent of covers if the Document is in electronic form.
Otherwise they must appear on printed covers that bracket the whole
aggregate.


\begin{center}
{\Large\bf 8. TRANSLATION\par}
\phantomsection
\addcontentsline{toc}{section}{8. TRANSLATION}
\end{center}


Translation is considered a kind of modification, so you may
distribute translations of the Document under the terms of section~4.
Replacing Invariant Sections with translations requires special
permission from their copyright holders, but you may include
translations of some or all Invariant Sections in addition to the
original versions of these Invariant Sections.  You may include a
translation of this License, and all the license notices in the
Document, and any Warranty Disclaimers, provided that you also include
the original English version of this License and the original versions
of those notices and disclaimers.  In case of a disagreement between
the translation and the original version of this License or a notice
or disclaimer, the original version will prevail.

If a section in the Document is Entitled ``Acknowledgements'',
``Dedications'', or ``History'', the requirement (section~4) to Preserve
its Title (section~1) will typically require changing the actual
title.


\begin{center}
{\Large\bf 9. TERMINATION\par}
\phantomsection
\addcontentsline{toc}{section}{9. TERMINATION}
\end{center}


You may not copy, modify, sublicense, or distribute the Document
except as expressly provided under this License.  Any attempt
otherwise to copy, modify, sublicense, or distribute it is void, and
will automatically terminate your rights under this License.

However, if you cease all violation of this License, then your license
from a particular copyright holder is reinstated (a) provisionally,
unless and until the copyright holder explicitly and finally
terminates your license, and (b) permanently, if the copyright holder
fails to notify you of the violation by some reasonable means prior to
60 days after the cessation.

Moreover, your license from a particular copyright holder is
reinstated permanently if the copyright holder notifies you of the
violation by some reasonable means, this is the first time you have
received notice of violation of this License (for any work) from that
copyright holder, and you cure the violation prior to 30 days after
your receipt of the notice.

Termination of your rights under this section does not terminate the
licenses of parties who have received copies or rights from you under
this License.  If your rights have been terminated and not permanently
reinstated, receipt of a copy of some or all of the same material does
not give you any rights to use it.


\begin{center}
{\Large\bf 10. FUTURE REVISIONS OF THIS LICENSE\par}
\phantomsection
\addcontentsline{toc}{section}{10. FUTURE REVISIONS OF THIS LICENSE}
\end{center}


The Free Software Foundation may publish new, revised versions
of the GNU Free Documentation License from time to time.  Such new
versions will be similar in spirit to the present version, but may
differ in detail to address new problems or concerns.  See
http://www.gnu.org/copyleft/.

Each version of the License is given a distinguishing version number.
If the Document specifies that a particular numbered version of this
License ``or any later version'' applies to it, you have the option of
following the terms and conditions either of that specified version or
of any later version that has been published (not as a draft) by the
Free Software Foundation.  If the Document does not specify a version
number of this License, you may choose any version ever published (not
as a draft) by the Free Software Foundation.  If the Document
specifies that a proxy can decide which future versions of this
License can be used, that proxy's public statement of acceptance of a
version permanently authorizes you to choose that version for the
Document.


\begin{center}
{\Large\bf 11. RELICENSING\par}
\phantomsection
\addcontentsline{toc}{section}{11. RELICENSING}
\end{center}


``Massive Multiauthor Collaboration Site'' (or ``MMC Site'') means any
World Wide Web server that publishes copyrightable works and also
provides prominent facilities for anybody to edit those works.  A
public wiki that anybody can edit is an example of such a server.  A
``Massive Multiauthor Collaboration'' (or ``MMC'') contained in the
site means any set of copyrightable works thus published on the MMC
site.

``CC-BY-SA'' means the Creative Commons Attribution-Share Alike 3.0
license published by Creative Commons Corporation, a not-for-profit
corporation with a principal place of business in San Francisco,
California, as well as future copyleft versions of that license
published by that same organization.

``Incorporate'' means to publish or republish a Document, in whole or
in part, as part of another Document.

An MMC is ``eligible for relicensing'' if it is licensed under this
License, and if all works that were first published under this License
somewhere other than this MMC, and subsequently incorporated in whole
or in part into the MMC, (1) had no cover texts or invariant sections,
and (2) were thus incorporated prior to November 1, 2008.

The operator of an MMC Site may republish an MMC contained in the site
under CC-BY-SA on the same site at any time before August 1, 2009,
provided the MMC is eligible for relicensing.


\begin{center}
{\Large\bf ADDENDUM: How to use this License for your documents\par}
\phantomsection
\addcontentsline{toc}{section}{ADDENDUM: How to use this License for your documents}
\end{center}

To use this License in a document you have written, include a copy of
the License in the document and put the following copyright and
license notices just after the title page:

\bigskip
\begin{quote}
    Copyright \copyright{}  YEAR  YOUR NAME.
    Permission is granted to copy, distribute and/or modify this document
    under the terms of the GNU Free Documentation License, Version 1.3
    or any later version published by the Free Software Foundation;
    with no Invariant Sections, no Front-Cover Texts, and no Back-Cover Texts.
    A copy of the license is included in the section entitled ``GNU
    Free Documentation License''.
\end{quote}
\bigskip
    
If you have Invariant Sections, Front-Cover Texts and Back-Cover Texts,
replace the ``with \dots\ Texts.'' line with this:

\bigskip
\begin{quote}
    with the Invariant Sections being LIST THEIR TITLES, with the
    Front-Cover Texts being LIST, and with the Back-Cover Texts being LIST.
\end{quote}
\bigskip
    
If you have Invariant Sections without Cover Texts, or some other
combination of the three, merge those two alternatives to suit the
situation.

If your document contains nontrivial examples of program code, we
recommend releasing these examples in parallel under your choice of
free software license, such as the GNU General Public License,
to permit their use in free software.

%---------------------------------------------------------------------
\end{document}


\end{document}

